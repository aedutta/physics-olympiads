\begin{solution}{normal}
% Since the velocity along the rod is zero, we can equate the projected component of velocities along the rod to get
% $$ - u\sin{\alpha} = v \cos{\alpha} \Rightarrow v = -u\tan{\alpha}$$
\begin{center}
    \begin{asy}
        unitsize(1cm);
        filldraw((0,6)--(0,0)--(4,0)--(4,-0.7)--(-0.7,-0.7)--(-0.7,6)--cycle,gray(0.75),invisible);
        draw((0,6)--(0,0)--(4,0),linewidth(1.5));
        draw((3,0)--(0,5),linewidth(2.5));
        draw(arc((3,0),0.7,180,120.96));
        label("$\alpha$",(2,0.5));
        draw((0,5)--(0,3.5),arrow=Arrow(),red+linewidth(0.7));
        draw((0,5)--(0,3.5),red+linewidth(1.5));
        label("$v$",(-0.3,4.25));
        draw((3,0)--(2,0),arrow=Arrow(),blue+linewidth(0.7));
        draw((3,0)--(2,0),blue+linewidth(1.5));
        label("$u$",(2.5,-0.3));
    \end{asy}
\end{center}

Since the length of the rod is constant, we can consider the equation
$$x^2+y^2 = L^2$$

Differentiating with respect to time, this gives
\begin{align*}
2x\frac{dx}{dt} + 2y\frac{dy}{dt} &= 0 \\
xu+yv &= 0 \\
u &= -v\frac{y}{x} = \boxed{-v\tan{\alpha}}
\end{align*}

Now, we find the acceleration
$$a = \frac{dv}{dt} = \frac{d(u\tan{\alpha})}{dt}$$

Since $u$ is constant, we have
$$a = u\frac{d\tan{\alpha}}{dt} = u \sec^2{\alpha} \frac{d\alpha}{dt}$$

The angular velocity $\dfrac{d\alpha}{dt}$ is simply
$$\frac{d\alpha}{dt} = \frac{u\cos{\alpha} + v\tan{\alpha} \sin{\alpha}}{L} = \frac{u}{L\cos{\alpha}}$$

This means that
$$a=u \sec^2{\alpha} \frac{d\alpha}{dt} = \boxed{\frac{u^2}{L\cos^3{\alpha}}}$$
\end{solution}
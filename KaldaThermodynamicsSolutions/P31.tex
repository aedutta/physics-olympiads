\begin{solution}{hard}
Since the potential energy in a sphere is lower, let us assume that the potential energy stored in the surface is going to completely transferred to kinetic energy instantaneously. Let us determine the radius of this sphere. Since the volume is equal, we have:
$$\pi(d/2)^2 h = \frac{4}{3}\pi r^3 \implies r = \frac{1}{2}\left(\frac{3d^2h}{2}\right)^{1/3}$$
Therefore, the change in energy is:
$$\sigma\Delta S = \sigma\pi dh - \sigma4\pi \left(\frac{1}{2}\left(\frac{3d^2h}{2}\right)^{1/3}\right)^2 = \sigma\pi h\sqrt[3]h\left(\sqrt[3]h-\sqrt[3]{9d/4}\right)$$
Let us assume this change in energy causes half of the liquid to move at a speed of $v$. Then:
$$\frac{1}{2}(0.5 m)v^2 = \sigma\pi h\sqrt[3]h\left(\sqrt[3]h-\sqrt[3]{9d/4}\right) \implies
v = 4\sqrt{\frac{\sigma}{\rho d}}\sqrt{\frac{\sqrt[3]h-\sqrt[3]{9d/4}}{\sqrt[3]h}}
$$
using the fact that $m=\rho \pi (d/2)^2h$. If we take $v$ to be the average velocity, then the characteristic time would be given by:
$$t=\frac{h}{v}=\frac{1}{4}\sqrt{\frac{\rho d}{\sigma}}\sqrt{\frac{\sqrt[3]{h^7}}{\sqrt[3]h-\sqrt[3]{9d/4}}}$$
We see that the characteristic time depends on the height $h$ of the original cylinder. Since we want the time for the most unstable perturbations, we want to maximize $t$ and we can do this by taking the derivative. Doing so gives us:
$$h=\left(\frac{7}{6}\right)^3\frac{9}{4}d\approx 3.57d$$
Therefore, the characteristic time is:
$$t=\frac{1}{4}\sqrt{\frac{\rho d}{\sigma}}\sqrt{\frac{\sqrt[3]{(3.57d)^7}}{\sqrt[3]{3.57d}-\sqrt[3]{9d/4}}} \approx 0.0088 \text{ s}$$
\tcbline 
\textbf{Solution 2.} Dimensional analysis tells us that the characteristic time is:
$$t =\sqrt \frac{\rho d^3}{\sigma}=0.003727 \text{ s}$$
\end{solution}
\begin{solution}{normal}
The total mass of the balloon is $Mg + m_{H_2} g$ which is equal to the mass of the air that is holding the balloon up $\rho_{\text{air}} Vg$. From the ideal gas law, we can write 
\[\rho_{\text{air}} = \frac{p\mu_{\text{air}}}{RT_{\text{air}}}\]
and similarly, we can write the mass of the hydrogen gas $m_{H_2}$ as 
\[m_{H_2} = \frac{p\mu_{H_2}V}{RT_{\text{air}}}.\]
We can then write that 
\[\rho_{\text{air}} Vg = Mg + m_{H_2} g\implies  \frac{p\mu_{\text{air}}}{RT_{\text{air}}}Vg = Mg + \frac{p\mu_{H_2}V}{RT_{\text{air}}}g.\]
This means that 
\[M_0 =  \frac{pV_0}{RT_{\text{air}}} (\mu_{\text{air}} - \mu_{H_2}).\]
The mass of the balloon when it reaches a volume $V_1$ is then given by 
\[M_1 =  \frac{pV_1}{RT_{\text{air}}} (\mu_{\text{air}} - \mu_{H_2})\]
and by Charle's law, note that 
\[\frac{V_1}{T_{\text{air}}} = \frac{V_0}{T_1}\]
which means that
\[M_1 =  \frac{pV_0}{RT_1} (\mu_{\text{air}} - \mu_{H_2}).\]
The ballast needed to be thrown out is then 
\[\Delta m = M_0 - M_1 = (\mu_{\text{air}} - \mu_{H_2})\frac{pV_0}{R}\left(\frac{1}{T_{\text{air}}} - \frac{1}{T_1}\right).\]
\end{solution}
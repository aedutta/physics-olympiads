\documentclass[a4paper,11pt]{article}
\usepackage[english]{babel}
\usepackage[a4paper,margin=0.6in]{geometry} % Page Dimensions

%%%%%%%%%%%%%%%%%%%%%%%%%%%%%%%%%%%%%%%%%%
%                PACKAGES                %  
%%%%%%%%%%%%%%%%%%%%%%%%%%%%%%%%%%%%%%%%%%

% Styling Choices
\setlength{\parskip}{\baselineskip}%

% Math
\usepackage{amsmath, amsthm, amssymb}
\usepackage[inline]{asymptote}

% Allows for hyperlinking
\usepackage{hyperref}
\hypersetup{
    colorlinks=true,
    linkcolor=magenta,
}

% Fancy Header
\usepackage{fancyhdr}
\pagestyle{fancy}
\lhead{Kalda Mechanics}
\rhead{\thepage}

% Coloured Boxes
\usepackage{xcolor}
\definecolor{border}{HTML}{004D4D}
\definecolor{hard}{HTML}{ffccb3}
\definecolor{easy}{HTML}{b3e6b3}
\definecolor{normal}{HTML}{f2f2f2}

% Syntax: \colorboxed[<color model>]{<color specification>}{<math formula>}
\newcommand*{\colorboxed}{}
\def\colorboxed#1#{%
  \colorboxedAux{#1}%
}
\newcommand*{\colorboxedAux}[3]{%
  % #1: optional argument for color model
  % #2: color specification
  % #3: formula
  \begingroup
    \colorlet{cb@saved}{.}%
    \color#1{#2}%
    \boxed{%
      \color{cb@saved}%
      #3%
    }%
  \endgroup
}

% Setup Gray Solution Boxes
\usepackage[breakable,many]{tcolorbox}
\newtcolorbox[auto counter]{solution}[1]{
    enhanced, breakable,
    arc=0pt,
    % colback=default, % Background color
    colframe=white, % Border Color
    coltitle=black, % Title Color
    fonttitle=\bfseries,
    title=\fcolorbox{border}{#1}{\textcolor{border}{pr} \bfseries \textcolor{border}{\thetcbcounter}.},
    attach title to upper,
    after title={\ },
    segmentation style={dashed, gray},
}

% Book's title and subtitle
\title{\Huge \textbf{Solutions to Problems in Relativity Handout by Siim Ainsar \footnote{All solutions will refer to the 1.2 English version given at \newline\url{https://www.ioc.ee/~kalda/ipho/meh_ENG2.pdf}}} \\ \huge With detailed diagrams and walkthroughs}
% Author
\author{\textsc{Ashmit Dutta, QiLin Xue, Kushal Thaman, Arhaan Ahmed}}

\begin{document}

\maketitle

%%%%%%%%%%%
% Preface %
%%%%%%%%%%%
\subsection*{Preface}
\vspace{-5mm}

This solutions manual came as a sort of pilot project on the online database \url{artofproblemsolving.com}. Kalda's problems never did have solutions written for them, thus the idea of creating solutions to these problems were an interesting idea. The majority of the solutions seen here were written on a private forum given to those who wanted to participate in making solutions. It was an interesting idea and the forum was able to have a great start. International users from India, the US, and Canada came into the forums to give ideas and methods to create what we see today.

\subsection*{Structure of The Solutions Manual}
\vspace{-5mm}
Each chapter in this solutions manual will be directed towards a section given in Kalda's mechanics handout. There are three major chapters: statics, dynamics, and revision problems. At the final page, we have an acknowledgments chapter. Each solution will be as detailed as possible and will usually contain a remark located in the footnote of each page. If you are stuck on a problem, and come here for reference, look at only the starting of the problem. Looking at the entire solution wastes the problem for you and ruins an opportunity for yourself to improve. Some solutions will have more than one solution to the problem.

\subsection*{Contact Us}
\vspace{-5mm}
Despite editing, there is almost zero probability that there are \textit{no} mistakes inside this book. If there are any mistakes amiss, you want to add a remark, have a unique solution, or know the source of a specific problem, then please contact us at \url{solutionstokaldahandouts@gmail.com}. Furthermore, please feel free to contact us at the same email if you are confused on a solution. Chances are that many others have the same confusion as you.
\newpage
\section{Solutions to Statics Problems}

This section will consist of the solutions to problems from problem 1-23 of the handout. Statics is typically the analysis of objects not in motion. However, objects travelling at constant velocity or with a uniform acceleration can be treated as a statics problem with a frame of reference change. This usually involves balancing forces, torques, and more to achieve equilibrium. 

\begin{solution}{normal}
Applying idea 1, we have:
$$P=C\frac{dT}{dt}$$
where
$$\frac{dT}{dt}=\frac{aT_0}{4}(1+a(t-t_0))^{-3/4}$$
Since we have that,
\[T(t) = T_0 [1 + a(t -t_0)]^{1/4}\]
dividing by $T_0$ gives us 
\[\frac{T}{T_0} = [1 + a(t - t_0)]^{1/4}.\]
We then get 
$$\frac{dT}{dt}=T_0^4(a/4)(1/T^3)$$
Plugging everything back in, we get:
$$\boxed{C=\frac{4PT^3}{aT_0^4}}$$
\end{solution}
\begin{solution}{normal}
The ice melts when the temperature of the kettle begins to drop. All the heat that was supplied to the kettle was used in melting the ice and bringing it up to the temperature of the rest of the water. Meanwhile the water already present lost some heat to the surroundings, and thus the graph dips at that point.
\vspace{3mm}

The total time for the temperature to recover to $T = 75^{\circ}\;\mathrm{C}$ is approximately $t = 37\;\mathrm{s}$. The heating rate of water is the slope of the graph or $dT/dt$. This means that the power at $T = 70^{\circ}\;\mathrm{C}$ is $P' = P \tan 75^{\circ}\approx 500\;\mathrm{W}$. Therefore, from energy conservation, we have that 
\[mL + mc\Delta T = Pt\implies m = \frac{Pt}{L + c\Delta T} \approx \boxed{28\;\mathrm{g}}.\]
\end{solution}
\begin{custom-simple}[Problem 3]
If the light is coherent, then the amplitude of the light emerging from each slit can individually be written as:
$$E=E_0\cos(\omega t + \phi)$$
where $\omega$ is the frequency and $\phi$ is the phase shift. An alternative way of writing this is:
$$E=E_0e^{i\omega t}e^{i\phi}$$
Its real component corresponds with the magnitude of the field that we measure at a certain time $t$. The complex number $e^{i\omega t}e^{i\phi}$ can be represented as a phasor which is essentially a vector with a constant magnitude of one that rotates in the complex plane at an angular frequency of $\omega$ (that is, it makes an angle $\omega t+\phi$ with the real axis). We want the sum of the three different amplitudes to sum up to zero, or:
$$E_1+E_2+E_3=E_0\cdot \left[e^{i\omega t}e^{i\phi_1}+e^{i\omega t}e^{i\phi_2}+e^{i\omega t}e^{i\phi_3}\right] = 0$$
Although this may look complicated, we can simplify it by treating them geometrically as phasors. To achieve an intensity minima of zero, we need three phasors such that their vector sum equals zero, which is equivalent to making a closed shape. Since they have the same magnitude, this shape must be an equilateral triangle. Without loss of generality, let $\phi_1=0$. This means $\phi_2=\phi_3=2\pi k + \frac{2\pi}{3}$ where $k$ is an integer.

The phase shift is defined as:
$$\frac{\phi}{2\pi} = \frac{(\text{path difference})}{\lambda} \implies k+ \frac{1}{3} = \frac{d\sin\theta}{\lambda}$$
where $d$ is the separation between two arbitrary slits and $\theta$ is the angle the light makes with the horizontal. Applying this for the slit separation distances in this problem, we have:
$$a = \left(k_1+\frac{1}{3}\right)\frac{\lambda}{\sin\theta_1}$$
$$b = \left(k_2+\frac{1}{3}\right)\frac{\lambda}{\sin\theta_2}$$
If we assume that $a,b \ll L$ where $L$ is the distance between the slits and the screen, then $\sin\theta_1=\sin\theta_2$. Taking the ratio, we get:
$$\frac{a}{b} = \frac{n}{m} \equiv \frac{3k_1+1}{3k_2+1}$$
This produces a minima of zero for any integer combinations of $k_1$ and $k_2$. We will prove that for each combination, $n-m$ is a multiple of three. We have:
$$3k_2+1 - (3k_2+1) = 3r \implies 3(k_2-k_1)=3r \implies k_2-k_1=r$$
where $r$ is an integer. Since $k_1$ and $k_2$ have to be integers, then $r$ must also be an integer, proving the statement.
\end{custom-simple}
\begin{solution}{easy}
Applying fact 6, we find that the initial heat flux is proportional to $\Delta T_1 = T_1 - T_0$ as the temperature change is minimal (in fact smaller than one Kelvin). Likewise, the final heat flux is proportional to $\Delta T_2 = T_2 - T_0$. Therefore, we have that the ratio of powers is \[\frac {P + P_{\text{man}}}{P} = \frac {\Delta T_2}{\Delta T_1}\]
Solving for $P_{\text{man}}$ and plugging in our numbers, we find that \[P_{\text{man}}= P\left(\frac {T_2-T_1}{T_1}\right) = \boxed{52.6\;\mathrm{W}}.\]
\end{solution}
\begin{solution}{easy}
\begin{center}
    \begin{asy}
    size(5cm);
draw((0,0)--(0,1)--(1,1)--(1,0)--cycle);
draw((0.5,0.5)--(0.25,0.5), arrow=Arrow(4));
label("$v$", (0.25, 0.5), W);
draw((0.5,0.5)--(0.5,0.75), arrow = Arrow(4));
label("$u$", (0.5, 0.75), E);
draw((0.5,0.5)--(0.75,0.25), arrow=Arrow(4));
label("$F_f$", (0.75, 0.25), SE);
    \end{asy}
\end{center}
In the board's frame of reference, there is only a horizontal force (the friction force), which has a constant direction that is anti-parallel to the velocity. Thus, the chalk moves in a $\boxed{\text{straight line}}$.
\end{solution}
\begin{solution}{hard}
a) We examine the forces involved in a cross-section of the cylinder. Assuming the block behaves like a point mass, and noting there is a centrifugal force, we create following diagram
\begin{center}
\begin{asy}
import olympiad;
size(5cm);
draw(circle((0,0), 1));
dot((sqrt(2)/2, sqrt(2)/2));
draw((sqrt(2)/2, sqrt(2)/2)--(0,0), arrow=Arrow(4));
draw((sqrt(2)/2, sqrt(2)/2)--(1.15, 1.15), arrow=Arrow(4));
draw((sqrt(2)/2, sqrt(2)/2)--(sqrt(2)/2, 0), arrow=Arrow(4));
draw((sqrt(2)/2, sqrt(2)/2)--(sqrt(2)/2+0.5, sqrt(2)/2-0.5), dotted, arrow=Arrow(4));
draw((sqrt(2)/2, sqrt(2)/2)--(sqrt(2)/2-0.5, sqrt(2)/2+0.5), arrow=Arrow(4));
draw((sqrt(2)/2, sqrt(2)/2)--(sqrt(2)/2, 0), dotted);
draw((0,0)--(sqrt(2)/2, 0), dotted);
draw((sqrt(2)/2, sqrt(2)/2)--(sqrt(2)/2+0.5, sqrt(2)/2), dotted);

pair A,B,C,D,E,F, G;
B = (sqrt(2)/2, sqrt(2)/2);
A = (sqrt(2)/2-0.5, sqrt(2)/2+0.5);
C = (sqrt(2)/2, 1.3);
D = (0, 0);
E = (sqrt(2)/2, 0);
F = (sqrt(2)/2+0.3, sqrt(2)/2);
G = (sqrt(2)/2+0.5, sqrt(2)/2-0.5);

draw(anglemark(G, B, F, 4));
draw(anglemark(E, D, B, 4));

label("$\theta$", (0.1, -0.05), NE);
label("$\theta$", (sqrt(2)/2+0.1, sqrt(2)/2+0.05), SE);
label("N", (0, 0), SW);
label("$\mu N$", (sqrt(2)/2-0.5, sqrt(2)/2+0.5), NW);
label("$mg$", (sqrt(2)/2, 0), S);
label("$m\omega^2 r$", (1.15, 1.15), NE);
\end{asy}
\end{center}
Because the system is in equilibrium we must set the resultant force to be zero in both directions. We assume a tilted coordinate of $\theta$ to perform our calculations on. In the vertical direction we have
\[0=N+mg\sin\theta-m\omega^2 r\]this in turn implies that the normal force is
\[N=m\omega^2 r-mg\sin\theta.\]Looking in the horizontal direction we note that
\[\mu N-mg\cos\theta=0\]\[mg\cos\theta=\mu N\]However, we remember that $\mu N$ is the maximum amount of friction obtained from slipping, thus we have to put a less than or equal sign to obtain
\[mg\cos\theta\leq\mu N\]substituting in $N$ from our previous calculation we have
\[mg\cos\theta\leq\mu(m\omega^2 r-mg\sin\theta)\]moving variables to the other side and canceling out $m$ gives
\[\omega^2 r\geq g(\mu^{-1}\cos\theta+\sin\theta)\]Our goal is to now to find a maximal value of $\mu^{-1}\cos\theta+\sin\theta$ on the interval $[0, 2\pi]$. It is known that a sinusoid $A\cos\theta+B\sin\theta$, can be represented as a single trigonometric function: $$A\cos\theta+B\sin\theta=\sqrt{A^2+B^2}\cdot\cos{(\theta + \phi)}$$
From these expressions of 1 sinusoid, it is clear the maximum value is $\sqrt{A^2+B^2}$, giving the maximum of $\mu^{-1}\cos\theta+\sin\theta$ as $\sqrt{1+\mu^{-2}}.$ Thus replacing this value in for our final expression gives us
\[\boxed{\omega^2 r\geq g(\sqrt{\mu^{-2}+1})}\]
b) In this part we work with cylindrical coordinates. We decompose gravity upon two axes. If we rotate the cylinder by $\alpha$ we have
\begin{align*}
g_{z}=g\sin\alpha\\ 
g_{r,\theta}=g\cos\alpha
\end{align*}All we do now is plug in $g_{\text{eff}}$ for our two equations. For our radial equation we had
\[N=m\omega^2 r-mg\sin\theta\]Since the normal force is radial we use $g_{r,\theta}=g\cos\alpha$ we plug in for gravity to get
\[N=m\omega^2 r-mg\cos\alpha\sin\theta\]In our second equation who have two components of gravity, $F_{\theta}$ and $F_z$, who’s combined modulus must be less than friction or $\mu N$.
\[\sqrt{F_{\theta}^2+F_r^2}\leq\mu N\]\[\sqrt{(mg\cos\alpha\cos\theta)^2+(mg\sin\alpha)^2}\leq\mu(m\omega^2 r-mg\cos\alpha\sin\theta)\]Taking out $m$ and factoring we have
\[\boxed{\omega^2 r\geq g\cos\alpha(\sqrt{\cos^2\theta+\tan^2\alpha}+\mu\sin\theta)}\]Again we must maximize our right hand equation. Inevitably, there is no neat trick to maximize this apart from differentiating and setting the result to zero.
\end{solution}

\begin{solution}{hard}
\begin{center}
    \begin{asy}
    size(8.5cm);
        import olympiad;
        pair A, B, C, D;
        A = (0,0);
        B = (2, 1);
        C = (2, 0);
        D = (2, -0.5);
        
        dot(A);
        draw(A -- B, arrow = Arrow(4));
        draw(A -- C, arrow = Arrow(4));
        draw("$v\cos\alpha$", A -- (1, 0), 2SE, arrow = Arrow(4));
        draw(B -- D, dashed);
        draw(anglemark(C, A, B));
        draw((2.6, 1) -- (2.6, -0.5), arrow=Arrow(4));
        
        /*labels and dots*/
        label("$\alpha$", (0.3, 0.01) , NE);
        label("$v$", B, N);
        label("$\frac{uL}{v\cos\alpha}$", (2, 0.6), E);
        label("$ut$", (2.6, 0.25), E);
        label("$s$", (2, -0.25), E);
    \end{asy}
\end{center}
Denote the dashed line by the wall which is a distance $L$ away from the point source.\vspace{3mm}

We express the lateral displacement of the ball as the sum of two components: lateral displacement in the air’s frame of reference, and the lateral displacement of the moving frame.\vspace{3mm}

In the air's frame the displacement is given by 
\[ut_{\text{air}} = \frac{uL}{v\cos\alpha}\]
and the lateral displacement in the moving frame is given by $s$. This gives us 
\[ut = \frac{uL}{v\cos\alpha} + s\implies t = \boxed{\frac{s}{u} + \frac{L}{v\cos\alpha}}\]
\end{solution}
\begin{solution}{hard}
\textbf{a)} Let the point where the rope meets the cylinder be $A$, and the two points where friction band meets the cylinder be $B$ and $C$. Let $D$ be the point diagonally opposite $A$.
\vspace{2mm}

\textbf{\textcolor[HTML]{3D85C6}{Claim.}} $D$ is the instantaneous centre of rotation (ICOR).
\begin{center}
\begin{asy}
size(5cm);
draw(circle((0,0), 1));
dot((0,1));
label("A", (0,1), N);
dot ((0,-1));
label("D", (0,-1), S);
draw((-sqrt(2)/2, sqrt(2)/2)--(sqrt(2)/2, sqrt(2)/2));
draw((-sqrt(2)/2-0.09, sqrt(2)/2-0.1)--(sqrt(2)/2+0.09, sqrt(2)/2-0.1));
label("B", (-sqrt(2)/2-0.09, sqrt(2)/2-0.1), NW);
label("C", (sqrt(2)/2+0.09, sqrt(2)/2-0.1), NE);
\end{asy}
\end{center}
\begin{proof} Let us assume a contradictory case. Let $D^*$ be the ICOR. Since the velocity of point $A$ is perpendicular to $AD$, $D^*$ must lie somewhere on $AD$. The velocities of $B$ and $C$ are perpendicular to $DB$ and $DC$ (due to definition of ICOR), and the friction forces are anti-parallel to these. The only forces acting on the cylinder is the tension $T$ due to the rope, and the two friction forces. As the cylinder is in equilibrium, by setting torque to be $0$ about the point where the two friction vectors intersect, we see that the tension vector must also pass through it. However, due to symmetry, the point of intersection must lie on $AD$ and thus it must be $A$ itself.
Thus, $\angle ABD^* = \angle ACD^* = 90^{\circ}$. Therefore this means that $ABCD^*$ is cyclic, which implies $D^* \equiv D$.
\end{proof}
\vspace{3mm}

Now let the angular velocity about $D$ be $\omega$. The velocity of $A$  is
$$v =\omega \times 2R$$
and the velocity at the centre is:
$$\boxed{v_\text{center}= \omega \times R = \frac{v}{2}}$$
\vspace{2mm}

\textbf{b)} Dividing the floor into various infinitesimally thin strips like in $a)$, we can conclude that the ICOR is still $D$ and the answer remains the same.
\end{solution}

\begin{solution}{normal}
Let us assume that $-dm'$ mass sublimes at some instant. ($m'$ is the total mass of the cup.)
Intially it is moving with $v$, the velocity of the vessel
Finally its velocity \textit{with respect to the vessel}, in the \textit{direction of motion} of the vessel becomes $-\frac{\sqrt {\frac{3RT}{\mu}}}{\sqrt 3}=-\sqrt{\frac{RT}{\mu}}.$\footnote{Since the rms speed is randomly directed in $x, y, z$ directions and we only want one speed, we divide by the magnitude of the unit vectors.}
Thus, it provides an impulse, $-(-dm' \Delta v )= -dm'\sqrt{\frac{RT}{\mu}}$ to the vessel
$$\therefore v = \int dv=\int_{M+m}^{M}\frac{-dm'\sqrt{\frac{RT}{\mu}}}{m'}=\sqrt{\frac{RT}{\mu}}\textup{ln}\left (1+\frac{m}{M}  \right )\approx \boxed{\sqrt{\frac{RT}{\mu}}\frac{m}{M}} $$
\end{solution}
\begin{solution}{normal}
\begin{center}
\begin{asy}
/* Geogebra to Asymptote conversion, documentation at artofproblemsolving.com/Wiki go to User:Azjps/geogebra */
import graph; size(5cm);
real labelscalefactor = 0.5; /* changes label-to-point distance */
pen dps = linewidth(0.7) + fontsize(10); defaultpen(dps); /* default pen style */
pen dotstyle = black; /* point style */
real xmin = -4.23213963380496, xmax = 7.371171204866221, ymin = -1.6441239198422954, ymax = 4.437435927619134; /* image dimensions */

/* draw figures */
draw((0,0)--(3,0), linetype("2 2"));
draw((3,0)--(3,4), linetype("2 2"));
draw((3,4)--(0,0), linetype("2 2"));
draw((0,0)--(2,0),EndArrow(6));
draw((2,0)--(2,2.7), EndArrow(6));
label("$\ell$",(1.37,2.6),SE*labelscalefactor);
label("$\mu N$",(0.823,0.4),SE*labelscalefactor);
label("$N$",(1.6,1.43),SE*labelscalefactor);
label("$h$",(3.15,2.22),SE*labelscalefactor);
label("$\sqrt{\ell^2-h^2}$",(1.2,-0.1),SE*labelscalefactor);
/* dots and labels */
dot((0,0),dotstyle);
label("$A$", (0,0), SW * labelscalefactor);
dot((3,0),dotstyle);
label("$B$", (3,0), NE * labelscalefactor);
dot((3,4),dotstyle);
label("$C$", (3,4), NE * labelscalefactor);
clip((xmin,ymin)--(xmin,ymax)--(xmax,ymax)--(xmax,ymin)--cycle);
/* end of picture */
\end{asy}
\end{center}
Consider what happens when the applied force approaches infinity. To maintain equilibrium, the friction force between the rod and the board must also increase. This friction force will also approach infinity. When dealing with large forces, we can ignore constant forces such as the weight of both the board and the rod.
\vspace{2mm}

As a result, since the weight of the rod is negligible we can pretend it's a mass-less rod. We also know that the forces at the ends of a massless rod will always point along the rod. For example, the force exerted on the rod by the board must point along the rod as well. The angle of this force is solely dependent on the friction coefficient $\mu_1$. Therefore:
$$\tan\alpha < \frac{\mu_1 N}{N} \implies \boxed{\mu_1>\frac{\sqrt{\ell^2-h^2}}{h}}$$
\tcbline
\textbf{Solution 2:} We want that when the board is on the verge of slipping then the rod should exert a larger force on the board (the rod should be pulled towards the board and not away from it). Consider the torque on the rod about the hinge point. We want that it should be clockwise when the block is on the verge of slipping.
\vspace{2mm}

Let the sum of normal reaction and friction force on the rod be $f$ (the normal points upwards and the friction points to the right). When the block is on the verge of slipping, the resultant makes an angle $\tan^{-1} \mu$ from the normal. We have:
$$\tau = mg\sin \alpha\frac{l}{2} + f \sin (\tan^{-1} \mu - \alpha)$$
considering clockwise torque to be positive. As the applied force on the block increases, $f$ also increases without bounds and because we want the torque to be clockwise no matter how much force we apply, the $mg$ term can be neglected. So
$$f \sin(\tan^{-1}\mu-\alpha) \ge 0$$
Since both $\tan^{-1}\mu$ and $\alpha$ are less than $90^{\circ}$, we can conclude that
$$\boxed{\tan^{-1} \mu \ge \alpha \implies \mu \ge \frac{\sqrt{l^2-h^2}}{h}}$$
\end{solution}

\begin{solution}{hard}
Let us divide the displacement into tiny pieces, $s=\sum\Delta s$ where $\Delta s = v\Delta t$. \vspace{3mm} 

If the function $v(t)$ were known, the last formula would have been completed our task, because $\sum v(t)\Delta t$ is the sum of rectangles making up the area under the $v-t$-graph. \vspace{3mm}

However, the acceleration is given to us as a function of $v$, hence we need to substitute $\Delta t$ with $\Delta v$. \vspace{3mm}

While trying to do that, we can introduce the acceleration (which is given to as a function of $v$):
\[\Delta t = \Delta v\cdot \frac{\Delta t}{\Delta v} = \frac{\Delta v}{\Delta v/\Delta t} = \frac{\Delta v}{a}.\]
In other words
\[s = \sum \frac{v}{a}\Delta v = \int \frac{v}{a(v)}dv.\]This tells us that the answer is equal to the area under a graph which depicts $\dfrac{v}{a(v)}$ as a function of $v$.\vspace{3mm}

Applying a \href{https://www.wolframalpha.com/input/?i=fit+\%7B0\%2C0\%7D\%2C\%7B2\%2C0.2\%7D\%2C\%7B3.1\%2C0.4\%7D\%2C\%7B4\%2C0.8\%7D\%2C\%7B3.3\%2C0.46\%7D\%2C\%7B2.3\%2C0.24\%7D\%2C\%7B2.4\%2C0.26\%7D\%2C\%7B0.7\%2C0.06\%7D\%2C\%7B1.7\%2C0.16\%7D\%2C\%7B0.5\%2C0.04\%7D}{quartic least-squares} fit to some of the discernible data points, we can see that the curve $a(v)$ is well approximated by the function $0.00617211 v^4 - 0.0301639 v^3 + 0.0581573 v^2 + 0.0546369 v + 0.000715828$.\vspace{3mm}

Taking the \href{https://www.wolframalpha.com/input/?i=integral+from+0+to+4+of+x\%2F\%280.00617211+x\%5E4+-+0.0301639+x\%5E3+\%2B+0.0581573+x\%5E2+\%2B+0.0546369+x+\%2B+0.000715828\%29}{integral}, we can see that
\begin{align*}
    \int_0^4\dfrac{v}{a(v)}dv&\approx\int_0^4\dfrac{v}{0.00617211 v^4 - 0.0301639 v^3 + 0.0581573 v^2 + 0.0546369 v + 0.000715828}dv\\
    &\approx \boxed{39\;\text{m}}
\end{align*}
\blfootnote{Don't worry if your answer isn't exactly the same as ours, as this result may be difficult to determine by hand with the graph provided. A rough approximation (within reasonable limits) would suffice.}
\end{solution}
\begin{solution}{normal}
First, we'll look at the behavior of the tension at the bottom. The vertical component of the tension has to support the weight of the block so we have:
$$2T_\text{bottom,y} = 2T_\text{bottom}\cos(\beta/2) = Mg$$The horizontal component is thus:
$$T_x=T_\text{bottom}\sin(\beta/2) = \frac{Mg}{2}\tan(\beta/2)$$Notice that this horizontal tension force will be constant in a massive rope. If we look at a differential area of the string, the only other force other than tension is the gravitational force downwards. To balance horizontal forces, the horizontal components of tension have to be constant. At the top of the rope, the vertical component of the tension has to support the weight of the block and the string. We have:
$$2T_\text{top,y} = 2T_\text{top}\sin\alpha = (M+m)g$$The horizontal component will thus be:
$$T_x = T_\text{top}\cos\alpha = \frac{(M+m)g}{2}\cot\alpha$$
Setting these two expressions for the horizontal tension equal gives:
$$M\tan(\beta/2)=(M+m)\cot\alpha \implies \boxed{\beta = 2 \tan^{-1}\left(\left(1+\frac{m}{M}\right)\cot \alpha\right)}$$
\end{solution}

\begin{solution}{normal}
The total mass of the balloon is $Mg + m_{H_2} g$ which is equal to the mass of the air that is holding the balloon up $\rho_{\text{air}} Vg$. From the ideal gas law, we can write 
\[\rho_{\text{air}} = \frac{p\mu_{\text{air}}}{RT_{\text{air}}}\]
and similarly, we can write the mass of the hydrogen gas $m_{H_2}$ as 
\[m_{H_2} = \frac{p\mu_{H_2}V}{RT_{\text{air}}}.\]
We can then write that 
\[\rho_{\text{air}} Vg = Mg + m_{H_2} g\implies  \frac{p\mu_{\text{air}}}{RT_{\text{air}}}Vg = Mg + \frac{p\mu_{H_2}V}{RT_{\text{air}}}g.\]
This means that 
\[M_0 =  \frac{pV_0}{RT_{\text{air}}} (\mu_{\text{air}} - \mu_{H_2}).\]
The mass of the balloon when it reaches a volume $V_1$ is then given by 
\[M_1 =  \frac{pV_1}{RT_{\text{air}}} (\mu_{\text{air}} - \mu_{H_2})\]
and by Charle's law, note that 
\[\frac{V_1}{T_{\text{air}}} = \frac{V_0}{T_1}\]
which means that
\[M_1 =  \frac{pV_0}{RT_1} (\mu_{\text{air}} - \mu_{H_2}).\]
The ballast needed to be thrown out is then 
\[\Delta m = M_0 - M_1 = (\mu_{\text{air}} - \mu_{H_2})\frac{pV_0}{R}\left(\frac{1}{T_{\text{air}}} - \frac{1}{T_1}\right).\]
\end{solution}
\begin{solution}{normal}
\begin{center}
    \begin{asy}
        unitsize(3cm);
        pair O = (0,0);
        pair A = (1,1);
        pair B = (2,-0.5);
        draw(O--A--A+B--B--O--B--B+(1,0.3)--A+B);
        draw(arc(B,0.3,16,45));
        label("$\alpha$",B+(0.38,0.22));
        draw(B/2-(1,0)--B/2,arrow=Arrow());
        draw(arc(B/2,0.5,180,166));
        label("$\beta$",B/2-(0.8,-0.08));
        draw(ellipse(B/2-(1.2,0.05),0.2,0.2/3));
        filldraw(B/2-(1,0.05)--B/2-(1.4,0.05)--B/2-(1.4,-0.05)--B/2-(1,-0.05)--cycle,gray(0.949),invisible);
        draw(ellipse(B/2-(1.2,-0.05),0.2,0.2/3));
        draw(B/2-(1,-0.05)--B/2-(1,0.05));
        draw(B/2-(1.4,-0.05)--B/2-(1.4,0.05));
        label("$v_0$",B/2+(-0.6,-0.1));
    \end{asy}
\end{center}

When on the plane, the puck experiences no change in its x-velocity, which is
$$v_0\cos\beta=5\;\text{m/s}$$

However, it experiences an acceleration parallel to the plane with magnitude
$$a=g\sin\alpha$$

We note from the trajectory given that the puck drops $2.5\;\text{m}$ below the apex of its trajectory while undergoing a horizontal displacement of $x=5\;\text{m}$.

The time it takes to complete this motion is
$$t=\dfrac{x}{v_0\cos\beta}=1\;\text{s}$$

Therefore, we have that
\begin{align*}
\dfrac{gt^2\sin\alpha}{2}&=2.5\\
\sin\alpha&=\dfrac{5}{gt^2}\\
\alpha&\approx \boxed{30\degree}
\end{align*}
\end{solution}
\begin{solution}{normal}
Due to symmetry the turtles meet at the centroid of the triangle formed, and form an equilateral triangle at any instant. The velocity of the first turtle with respect to the other is obviously $$0.1 \cos\left(60\degree\right) \;\text{m/s}$$

Thus, the relative velocity of separation is
$$v(1+\cos\left(60\degree\right)) = \frac{3v}{2}$$

Since this is constant, the time taken for the turtles to meet is
$$t = \frac{d}{\tfrac{3v}{2}} = \frac{2d}{3v} = \boxed{6.7\;\text{s}}$$

\tcbline

\textbf{Solution 2:} The path length of any turtle in the motion is simply
$$ds = dr\sqrt{1+\left(\frac{rd\theta}{dr}^2\right)}$$

Using polar coordinates, one can deduce that
\begin{align*}
\frac{dr}{dt} &= -10\sin\left(60\degree\right)\\
v\frac{d\theta}{dt} &= \frac{10}{2}=5
\end{align*}

Hence we have
$$\frac{d\theta}{dr} = -\frac{1}{r\sqrt{3}} \Rightarrow ds = \frac{2dr}{\sqrt{3}}$$

Since $t = \dfrac{ds}{10} $, we find the total time by integrating this expression:
\begin{align*}
\int_{0}^{T}{dt} &= -\frac{2}{\sqrt{3}v}\int_{\frac{1}{\sqrt{3}}}^{0}{dr}\\
T &=  \frac{2d}{3v} = \boxed{6.7\;\text{s}}
\end{align*}
\end{solution}
\begin{custom-simple}[Problem 16]
\begin{enumerate}
\item There is no light coming out from outlet $C_2$ because at the junction point a wave is generated in upper fiber in the same direction as the circular fiber (Huygens principle can be used to prove this).As energy at steady state is constant we can say that the total energy input $=$ total energy output, hence $I_{A_2}+I_{C_1}=I_0$. The result is a mirror image of the graph in the problem text that touches $I = 0$ at the bottom and $I = I_0$ at the top.
\item At this wavelength, all intensity $I_0$ comes out from fiber $C_1$ and intensity in fiber $B$ and intensity in direction $C_1$ should have ratio $(1-\alpha)/\alpha$. So $$I_B = \frac{I_0(1-\alpha)}{\alpha} = 99I_0$$
\item The intensity of light in fiber B is maximum when the light circulating in the fiber reaches the lower junction point in the same phase as the light from fiber A. Then the intensity going to fiber C is also maximum. Thus, fiber B must accommodate an integer of n wavelengths. From the graph we see that two successive resonances occur at wavelengths $\lambda_0 = 1660$ nm and $\lambda_1 = 1680$ nm. So $n\lambda_1 ’ = (n+1) \lambda_0 ’= l$, where $l$ is the desired length and the second resonant wavelength in the fiber is $\lambda_1 ’ = \lambda_0 ’\frac{\lambda_1 }{\lambda_0}$. From this relation we find $\frac{1}{n}= \frac{\lambda_1 ’}{\lambda_0 ’} - 1$ and
$$ l = \frac {\lambda_0 ’\lambda_1}{\lambda_1-\lambda_0} = \boxed{84\;\mathrm{\mu m}}$$
\end{enumerate}
\end{custom-simple}
\begin{solution}{normal}
From the prelude, we have the work as:
$$W = \frac{m}{2}(v_2^2-v_1^2) + \Delta U$$
The internal energy changes both in gravitational energy and in internal energy so:
\begin{align*}
\Delta U &= mg\Delta h + nC_v\Delta T \\
&= mg\Delta h + \left(nR\Delta T\right)\frac{C_v}{R} \\
&= mg\Delta h + \Delta(PV)\frac{C_v}{R} \\
\end{align*}
The compression work $W$ is:
$$W=-\Delta(PV)$$
so putting everything together gives:
$$0=\frac{m}{2}(v_2^2-v_1^2)+mg\Delta h + \Delta(PV) \left(\frac{C_v+R}{R}\right)$$
Having $C_P=C_V+R$, $\Delta(PV)=nR\Delta T$, and letting the molar mass $\mu\equiv m/n$, we can simplify this to: 
$$0=\frac{1}{2}(v_2^2-v_1^2)+g(h_2-h_1) + \frac{C_P(T_2-T_1)}{\mu}\implies \frac{1}{2}v^2 + gh + c_p T = 0$$
\end{solution}
\begin{custom-simple}[Problem 18]
\textbf{(a)} By energy conservation, the amplitudes of the output wave and input wave must be the same. The output fiber wave is formed by the sum of the wave in the fiber and the wave from the other fiber. According to the energy conservation, the amplitude of each component is $\sqrt 2$ times smaller than the original when the wave enters only one fiber. Thus, while the amplitude of the incoming waves was A, the outgoing resultant wave is in an expressible form.
$$A = \sqrt {\left(\frac {A}{\sqrt 2}\right)^2 \cdot 2 + 2\left(\frac {A}{\sqrt 2}\right) \left(\frac {A}{\sqrt 2}\right)\cos \phi}$$where $\phi$ is the phase shift. So $\cos (\phi/ 2) = 1/\sqrt 2$ and consequently $\phi = \frac {\pi}{2}$
\vspace{5mm}

\textbf{(b)} Phase difference between the $2$ fibers is $\pi$, the minima condition in fiber $1$ is $\Delta l = n\lambda$, where n is an integer. Writing this as $n = \frac{\Delta l}{\lambda}$ we see that
\[\frac{\Delta l}{\lambda_{\text{min}}}\geq n \geq \frac{\Delta l}{\lambda_{\text{max}}}\]thus $49.2 \geq n \geq 45.4$ and the values of $n$ to be sought are $46, 47, 48$ and $49$. The corresponding wavelengths are given by the formula $\lambda = \frac{n}{\Delta l}$; these are $612, 625, 638$ and $652\;\mathrm{nm}$
\end{custom-simple}
\begin{solution}{normal}
Consider a horizontal displacement of a gas. The impulse the gas experiences is:
$$J = F_\text{net} dt = \Delta p S dt = (p_1-p_2)Sdt$$
where $S$ is the cross sectional area. Take $p_1 > p_2$ such that the gas is moving towards the right (which we arbitrarily set as the positive direction). This gives the change in momentum to be:
$$\Delta p =(\rho Sx_2)v_2-(\rho Sx_1)v_1 =(\rho S)v_2^2 dt-(\rho Sx_1)v_1^2 dt$$
Setting these two expressions equal gives:
$$p_1-p_2=\rho v_2^2 - \rho v_1^2$$
Since this is true for any two intervals, then the quantity
$$p+\rho v^2 = \text{constant}$$
must be preserved.
\end{solution}
\begin{solution}{normal}
Let us consider a sound wave which propagates in the direction of $x$-axis; then, the air density $\rho = \rho (x - c_s t)$. Following the idea 16, we consider a frame which moves with speed $c_s$, with coordinate axis $x' = x - c_s t$. In that frame, the density perturbation remains constant in time, $\rho = \rho (x')$. This means that we can use idea 14, so that we obtain two equations:
\begin{align*}
(\rho + \Delta \rho) (v + c_s) &= \rho c_s\\
\frac{1}{2} (v + c_s)^2 + c_p T &= \frac{1}{2}c_s^2 + c_p T
\end{align*}
where $v\ll c_s$ is the speed of the gas in the laboratory frame. Note that we also have our momentum equation to be 
\[P + (\rho_0 + \Delta \rho)(v + c_s)^2 + c_p T = P_0 + \rho_0 c_s^2.\]
From the problem statement, we have that $\Delta \rho_0 = \rho_0$ which means that from our first equation we have 
\[2\rho_0 (v + c_s) = \rho_0 c_s\implies v = -\frac{c_s}{2}.\]
Going into our momentum equation, we have that 
\[P + 2\rho_0 \left(c_s - \frac{c_s}{2}\right)^2 = P_0 + \rho_0 c_s^2\implies P = P_0 + \frac{\rho_0 c_s^2}{2}.\]
Also, note that by the ideal gas law, we have:
$$PV=nRT \implies P\mu=\rho RT\implies T = \frac{P\mu}{\rho R}.$$
We can now use these results to go into our second equation (energy) to find that 
\[\frac{1}{2}\left(c_s - \frac{c_s}{2}\right)^2 + c_p \left(\frac{P_0 + \rho_0 c_s^2/2}{2R\rho}\right) = \frac{1}{2}c_s^2 + c_p T.\]
We can simplify further to get 
\[\frac{1}{8}c_s^2 + c_p \left(\frac{P_0\mu}{2R\rho_0} + \frac{c_s^2 \mu}{4R}\right) = \frac{1}{2}c_s^2 + c_p T\]
then expanding tells us 
\[\frac{c_s^2}{8} + \frac{c_p T_0}{2} + \frac{c_p c_s^2\mu}{4R} = \frac{1}{2}c_s^2 + c_p T.\]
From the problem, we see that $c_p = 5R/2\mu$ and solving gives us the equation 
\[c_s = \sqrt{\frac{5RT_0}{2\mu}} = \sqrt{\frac{5}{2}\left(\frac{\gamma RT_0}{\mu}\right)} \implies \text{the speed of the wave becomes } \sqrt{2.5} \text{ times faster.}\]
\end{solution}
\begin{solution}{hard}
First, we make the following claim: \vspace{3mm}

\boldclaim{Claim:} The optimal-velocity trajectory must contain both endpoints of the roof along its path.
\begin{center}
    \rule{7cm}{0.4pt}
\end{center}

\textit{Proof:}
Assume for the sake of contradiction that the optimal-velocity trajectory hits neither one of the two endpoints of the roof. Then, we can clearly see that reducing the velocity by an small amount would still result in the stone clearing the roof.\vspace{3mm}

Now assume that the optimal-velocity trajectory hits only one of the two endpoints of the roof. In both cases, the thrower can displace themself horizonally by an small amount, resulting the stone hitting neither one of the two endpoints.\vspace{3mm}

Thus, the optimal-velocity trajectory must contain both endpoints of the roof.
\begin{center}
    \rule{7cm}{0.4pt}
\end{center}

By idea 28, we can set the rightmost point of the roof (point $F$) to be the focus of the region $\mathcal{R}$ of all possible trajectories. Optimally, this parabola should pass through the left end of the roof.

\begin{center}
    \begin{asy}
        import graph;
        unitsize(8mm);
        filldraw((-7,0)--(-7,-0.6)--(12.2,-0.6)--(12.2,0)--cycle,gray(0.6),invisible);
        filldraw((2,0)--(2,6.8)--(9.6,4.7)--(9.6,0)--cycle,gray(0.8));
        draw((-7,0)--(12.2,0),linewidth(2));
        draw((2,0)--(2,6.8)--(9.6,4.7)--(9.6,0),linewidth(2));
        draw((0.7,7.2)--(10.7,4.4),linewidth(4));
        draw((0.6,0)--(0.6,7.2),Arrows);
        draw((10.8,0)--(10.8,4.4),Arrows);
        draw((0.7,7.5)--(10.7,4.7),Arrows,Bars);
        label("$a$",(0.3,3.6));
        label("$b$",(5.7,6.5));
        label("$c$",(11.1,2.2));
        pair f(real x){
        	return (x,-0.0379*x^2+0.81106*x+6.65083);
        }
        draw(graph(f,0,13.7),red);
        pair g(real x){
        	return (x,-0.101*x^2+0.87*x+6.639);
        }
        draw(graph(f,-6.3,13.7),red);
        draw(graph(g,-4.9,12.5),blue);
        draw((10.7,4.4)--(10.7,4.4)+(1.5,-1.5*1.29),arrow=Arrow());
        label("$v$",(11.8,3.5));
        draw((10.8,4.5)--(10.8,10.95),Arrows,Bars);
        label("$h$",(11.1,7.8));
        label("$F$",(11.2,4.5));
        draw((-4.9,0)--(-4.9,0)+(2,2*1.86),arrow=Arrow());
        label("$v_0$",(-3.5,1.5));
    \end{asy}
\end{center}

By fact 9, we have that
$$h=\dfrac{a+b-c}{2}$$

We know that if the projectile is thrown straight up, it hits the top of the red parabola, so
\begin{align*}
\dfrac{1}{2}v^2&=gh\\
v&=\sqrt{g(a+b-c)}
\end{align*}

By idea 32, we have that
\begin{align*}
\dfrac{1}{2}v^2+gh&=\text{constant}\\
\dfrac{1}{2}v_0^2&=\dfrac{1}{2}v^2+gc\\
v_0&=v_{\text{min}}=\boxed{\sqrt{g\left(a+b+c\right)}}
\end{align*}
\tcbline
\textbf{Solution 2:}
We begin this solution by also proving that the optimal-velocity trajectory must pass through the two endpoints of the roof.\vspace{3mm}

Then, we set of coordinates of $F$ to be $(0,0)$, so the coordinates of the left end of the roof are $\left(\sqrt{b^2-(a-c)^2},a-c\right)$, where we have taken the absolute value of the x-coordinate to make calculations easier.\vspace{3mm}

Let $\theta=\arctan\left(\dfrac{a-c}{\sqrt{b^2-(a-c)^2}}\right)$, let the initial launch angle (to the horizontal) be $\alpha$, and let the initial velocity of the stone be $v_0$.

\begin{center}
    \begin{asy}
        unitsize(3cm);
        draw((0,0)--(1/sin(25*pi/180),0)--(1/sin(25*pi/180),1)--cycle);
        draw(arc((0,0),0.4,0,24));
        label("$\theta$",(0.5,0.1));
        label("$b$",(1.2,0.65));
        label("$a-c$",(2.6,0.5));
        label("$\sqrt{b^2-(a-c)^2}$",(1.2,-0.2));
    \end{asy}
\end{center}

The equation for the slope of the roof is given by
$$y=x\tan\theta$$

Along the slope of the roof, we have that
\begin{align*}
x&=v_0t\cos\alpha\\
y&=v_0\sin\alpha t-\dfrac{gt^2}{2}\\
v_0\sin\alpha t-\dfrac{gt^2}{2}&=v_0t\cos\alpha\tan\theta\\
t&=\dfrac{2v}{g}\left(\sin\alpha-\cos\alpha\tan\theta\right)\\
\end{align*}

It suffices to maximize the horizontal distance travelled, which is
$$x=\dfrac{2v^2\cos\alpha}{g}\left(\sin\alpha-\cos\alpha\tan\theta\right)$$

Taking the derivative with respect to $\alpha$, we get that
\begin{align*}
\dfrac{dx}{d\alpha}&=\dfrac{2v^2}{g}\left(2\tan\theta\sin\alpha\cos\alpha-\sin^2\alpha+\cos^2\alpha\right)\\
&=\dfrac{2v^2}{g\cos\theta}\cos\left(\theta-2\alpha\right)\\
\implies\alpha&=\dfrac{\pi}{4}+\dfrac{\theta}{2}
\end{align*}

From this, we can find that
\begin{align*}
\sin\alpha&=\sqrt{\dfrac{a+b-c}{2b}}\\
\cos\alpha&=\sqrt{\dfrac{-a+b+c}{2b}}\\
\cos\theta&=\dfrac{\sqrt{b^2-(a-c)^2}}{b}\\
\tan\theta&=\dfrac{a-c}{\sqrt{b^2-(a-c)^2}}
\end{align*}

Then, we must have that
\begin{align*}
vt\cos\alpha&=b\cos\theta\\
\dfrac{2v^2\cos\alpha\left(\sin\alpha-\cos\alpha\tan\theta\right)}{g}&=b\cos\theta\\
v&=\sqrt{\dfrac{gb\cos\theta}{2\cos\alpha(\sin\alpha-\cos\alpha\tan\theta)}}
\end{align*}

Plugging everything in, this simplifies (quite miraculously) to
$$v=\sqrt{g(a+b-c)}$$

Applying conservation of energy to find the velocity at the ground, we see that
$$\boxed{v_0=\sqrt{g(a+b+c)}}$$
\end{solution}
\begin{solution}{normal}
Consider a vertical plane parallel to the free hanging portion of the string. 
\begin{center}
\begin{asy}
import graph; size(8cm);
real labelscalefactor = 0.5; /* changes label-to-point distance */
pen dps = linewidth(0.7) + fontsize(10); defaultpen(dps); /* default pen style */
pen dotstyle = black; /* point style */
real xmin = -4.773372331294487, xmax = 6.660264590692777, ymin = -0.8080629936799328, ymax = 5.184566884291457; /* image dimensions */

/* draw figures */
draw((-1,1)--(-2,3)--(2,5)--(3,3));
draw((-1,1)--(3,3));
draw((1.5,3.5)--(1.5,0));
draw((1.5,3.5)--(2,2.5));
draw((-1,1)--(0,1), linetype("2 2"));
draw(shift((-1,1))*xscale(0.5)*yscale(0.5)*arc((0,0),1,0,26.565051177077994));
draw(shift((1.5,3.5))*xscale(0.3253809639500407)*yscale(0.3253809639500407)*arc((0,0),1,270,296.565051177078));
label("$\frac{\pi R}{2}$",(1.7792383023590328,3.3),SE*labelscalefactor);
label("$L$",(1.3,1.8),SE*labelscalefactor);
label("$\alpha$",(-0.48575848554926304,1.2),SE*labelscalefactor);
label("$\alpha$",(1.5,3.1),SE*labelscalefactor);
label("$\ell$",(1.2944788422162978,2.9),SE*labelscalefactor);
/* dots and labels */
dot((1.5,3.5),linewidth(4pt) + dotstyle);
label("$P$", (1.5368585722876653,3.5631300693312764), NE * labelscalefactor);
dot((1.5,2.25),linewidth(4pt) + dotstyle);
label("$Q$", (1.5368585722876653,2.2), SE * labelscalefactor);
clip((xmin,ymin)--(xmin,ymax)--(xmax,ymax)--(xmax,ymin)--cycle);
/* end of picture */
\end{asy}
\end{center}
Move this plane until it contacts the point in which the cylinder and the string first meet. Call this point $Q$, which is where we unfold half of the cylinder into a rectangle where the width is $\pi R$. The angle between $PQ$ and the width of the rectangle is $\alpha$ so we have:
$$PQ = \frac{\pi R}{2\cos\alpha}$$and thus:
$$\boxed{\ell = L - \frac{\pi R}{2\cos\alpha}}$$When the weight oscillates, the trace of the string still stays straight on the unfolded cylinder. Therefore the length of the hanging string
(and thus the weight’s potential energy) do not depend in any oscillatory state on whether the surface of the cylinder is truly
cylindrical or is unfolded into a planar vertical surface. Therefore the period of oscillations is still
$$\boxed{T=2\pi\sqrt{L/g}}$$
\end{solution}

\begin{solution}{normal}
Label the strings from left to right as $1$, $2$, $3$, $4$. If string $4$ is cut then in equilibrium state:
$$T_1+T_2+T_3 = mg$$Let the rod be inclined at an angle $\theta$ with the horizontal in equilibrium position. As extensions in the strings will be small $\theta$ will be very small. Balancing torques about $1$, we get:
$$
T_2\left(\frac{\ell}{3}\right) + T_3\left(\frac{2\ell}{3}\cos\theta\right) =
mg\frac{\ell}{2}\cos\theta \implies T_2 + 2T_3
= \frac{3mg}{2}
$$As the rod is rigid, we can write our third equation as a conservation law:
$$\frac{\Delta x_2 - \Delta x_1}{\ell/3} = \frac{\Delta x_3 - \Delta x_1}{2\ell/3} \implies \frac{T_2-T_1}{k\ell/3} = \frac{T_3-T_2}{k\ell/3}$$As strings are identical:
$$2T_2 = T_3+T_1$$We have three equations and three unknowns so solving them yields:
\begin{align*}
T_1 &= \frac{1}{12} mg \\
T_2 &= \frac{1}{3} mg \\
T_3 &= \frac{7}{12} mg \\
\end{align*}
\end{solution}

\begin{solution}{normal}
\textbf{a)} We are given that$$\ln \frac{p_i}{p_0} = \frac{a_i}{T} + b_i$$Substittuing the values from the table for $A$ we get,
$$\ln 0.284 = \frac{a_A}{313} + b_{A}$$and$$\ln 1.476 = \frac{a_A}{363} + b_{A}.$$
Solving these we get $a_A = -3748.49K$ and $b_A = 10.72$.
The boiling temperature is the temperature at which the saturated vapour pressure equals the atmospheric pressure, i.e. $\frac{p_A}{p_0} = 1$
This gives$$\ln 1 = \frac{a_A}{T_{A}} + b_A \implies \boxed{T_{A} = \frac{a_A}{b_A}\approx 350K}$$Similarly solving for $B$ gives,
$$a_B = -5121.64$$$$b_B = 13.735$$$$\boxed{T_B \approx 373K}$$
\textbf{b)} Firstly, evaporation will start at the interface.
We use fact 15 to conclude that $\frac{p_A}{p_0} + \frac{p_B}{p_0} = 1$, at the time $t_1$
$$\implies e^{\frac{a_A}{t_1} + b_A}+e^{\frac{a_B}{t_1} + b_B}-1 =0.$$
Bisection method (or you could just randomly put some values less than 370 K, to zero in on the root) can be used to find the root of this equation, which gives $\boxed{t_1 \approx340 K =  67^{\circ}}$
The saturated vapour pressures for the two liquids at his temperature are$$p_A \approx 0.734 p_0$$$$p_B \approx 0.267p_0$$.

Now let $m_A$ and $m_B$ be the mass of liquid $A$ and $B$ that escape in a bubble. We have $\frac{m_A}{\rho_A} = \frac{m_B}{\rho_B} \implies\frac{m_A}{m_B} =  \frac{p_AM_A}{p_BM_B} \implies \frac{m_A}{m_B} = 22$. Here $\rho_A$ and $\rho_B$ are the densities of the vapours of $A$ and $B$.
Thus the rate at which $A$ evaporates is $22$ times that of $B$.
Therefore the temperature starts increasing again when $A$ is completely evaporated. The amount of $B$ that has evaporated during this time is $\frac{100}{22}= \ 4.5g$.
Thus at $\tau_1$, there will be no $A$ left, while $95.56g$ of $B$ will be left.
Also, the temperature $t_2$ is the boiling point of $B$.
$$\boxed{t_2 = 373K=100^{\circ} \text{ C}}$$
\end{solution}
\begin{solution}{normal}
Kinematics tells us that
\[s = vt\]Using fact 1, we find that
\[\boxed{s = \gamma v\tau}\]
\end{solution}

\end{document}

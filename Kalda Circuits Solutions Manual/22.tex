\begin{solution}{hard}
We use the idea of "negative resistance". In this case, the edge connecting vertices $\text{A}$ and $\text{B}$ is cut off. We can represent the missing wire as a parallel connection of $R$ and $-R$. This is because idea 22 tells us that a resistor and a negative resistor in parallel corresponds to infinite resistance which is what $\text{AB}$ essentially is since no current can flow through. From problem 21, we note that
\[R = \frac{19}{30}R\]and since we have a $R$ and $-R$ resistor in parallel (as shown below)
\begin{center}
\begin{asy}
size(6cm);
draw((0,0) -- (0.5, 0));
dot((0.5, 0));
draw((0.5, 0) -- (0.5, 0.25));
draw((0.5, 0) -- (0.5, -0.25));
draw((0.5, 0.25) -- (1.5, 0.25), purple + blue);
label("$R$", (1, 0.25), N);
draw((0.5, -0.25) -- (1.5, -0.25), purple + red);
draw((1.5, -0.25) -- (1.5, 0.25));
dot((1.5, 0));
draw((1.5, 0) -- (2, 0));
label("$-R$", (1, -0.25), S);
label("A", (0.5, 0), NW);
label("B", (1.5, 0), NE);
\end{asy}
\end{center}
We can reduce the equivalent resistance by idea 1:
\[\frac{1}{R_{\text{eq}}} = \sum_{i} \frac{1}{R_{i}} = \frac{1}{R} + \frac{1}{-R}\implies R_{\text{eq}} = \frac{(\frac{19}{30}R)(-R)}{\frac{19}{30}R - R} = \frac{19}{11}R.\]
\end{solution}
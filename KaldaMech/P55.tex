\begin{solution}{normal}\textbf{a)} We can draw an analogy with thermodynamics. Since the process is slow, and there are no external work being done, we can say that the process is adiabatic, that is:
$$TV^{\gamma-1}=\text{constant}$$Here,
$$\gamma = \frac{1+2}{1} = 3$$since there is only one degree of freedom. Therefore, when the distance doubles, the volume doubles and $V^{\gamma-1}$ will quadruple. As a result, $T$ will decrease by a factor of four. However, since $T\propto v^2$, the speed must decrease by a factor of two. Therefore,
$$v=\boxed{5\text{ m/s}}$$
\textbf{b)} The average force corresponds with the pressure. We have:
$$PV^\gamma=\text{constant}$$Since the $V^\gamma$ term increases by a factor of eight, pressure, and thus the average force will decrease by a factor of eight.
\tcbline
\textbf{Solution 2:} We can analyze the adiabatic invariant of the system. Let us draw a phase diagram of the entire system.
\begin{center}
\begin{asy}
import graph;
size(5cm);

draw((0, 0)--(1.5,0), Arrow(TeXHead));
label("$x$",(1.5,0),S);
draw((0,-1.5)--(0,1.5), Arrows(TeXHead));
label("$p(x)$",(0,1.5),E);

draw((0, 0.75) -- (1, 0.75) -- (1, -0.75) -- (0, -0.75), red);
label("$mv$", (1, 0.4), E);
label("$L$", (0.4, 1.02));
\end{asy}
\end{center}
From here, we see that the adiabatic invariant of the system is given by
\[I = 2mvL\]where $v$ and $L$ can change. Thus, the initial adiabatic invariant, $I_0$, is given by $2mv_0 L$, and the final adiabatic invariant, $I_f$, is given by $2mv(2L) = 4mvL$. This means that
\begin{align*}
I_0 = I_f \implies 2mv_0 L &= 4mvL\\
v &=\frac{1}{2}v_0 = \boxed{5\;\mathrm{m/s}}
\end{align*}
\end{solution}

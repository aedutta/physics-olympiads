\begin{solution}{hard}
Since the second block is being pushed rightwards with some velocity it is in turn pulling more string outwards. By conservation of string, the block that isn’t pushed will be pushed upwards because less string is there to sustain it’s mass. Thus the answer is that the block that isn’t pushed will reach higher after subsequent motion.

\tcbline

Let the tension in the string be $T$. Then at a certain instant, when the angle between the right mass and the vertical is $\alpha$, we have the component of vertical force to be 
\[F_{y_1} =  mg - T\cos\alpha.\]
At the other end, we have the component of vertical force to be 
\[F_{y_2} = mg - T.\]
Comparing the two accelerations at both ends gives us 
\[a_2 - a_1 = \left(g - \frac{T}{m}\right) - \left(g - \frac{T}{m}\cos\alpha\right) = \frac{T}{m}(1 - \cos\alpha)\]
which is always a non negative number. This implies that at any instant, the right load is lower than the left load.
\blfootnote{This problem was found in the book ’Aptitude Test Problems in Physics’ by S.S. Krotov though in that problem, only velocity was asked for.}
\end{solution}

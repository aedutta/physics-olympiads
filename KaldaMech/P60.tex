\begin{solution}{normal}
\begin{center}
\begin{asy}
size(10cm);
draw(arc((0,0),(1,0), (-1,0)));
draw((1,0)--(-1,0));
pair A,B,C,D;
A = (3.1610980056834	,	-0.086367809191);
B = (-2.4770577190321	,	1.965753050762);
C = (3.5031181490091	,	0.8533248115948);
D = (-2.1350375757064	,	2.9054456715488);
draw(A--B);
draw(B--D);
draw(C--D);
draw(C--A);
pair P,L, CM;
P = (0.34202014332567, 0.93969262078591);
L=(0,1.06417777248);
CM = L+P/2;
draw((0,0)--P, dotted);
draw((0,0)--(0,1.7),dotted);
draw(L--CM, dotted);
label("C", CM, E);
dot(CM);
label("P", P, NE);
dot(P);
label("L", L, NW);
dot(L);
markscalefactor=0.1;
draw(anglemark(P, (0,0), L, 4));
markscalefactor=0.05;

draw(anglemark(CM, L, (0,2), 4));

label("$\theta$",(0,0.4), NE);
label("$\theta$",L+(-0.05,0.2), NE);
label("O", (0,0), S);
\end{asy}
\end{center}
Here $P$ is the point of contact of the plank with the hemisphere after turning through $\theta$ and $L$ is the original contact point. $C$ is the centre of mass of the plank.
\vspace{2mm}

To solve the problem, let us turn the plank through an angle of $\theta$ and consider the torques at that moment. First of all, we see that
$$\angle POL = 90^{\circ} - \angle OLP = \theta$$
We want $C$ to generate a clockwise torque about $P$, so the distance between $C$ and $OL$ must be greater than the distance between $P$ and $OL$. This means
$$\frac{h}{2} \sin \theta <R \sin \theta \implies \boxed {R>\frac{h}{2}}$$

\tcbline
If the initial position was stable, then slight deviations would cause the center of mass to be higher. Therefore, we want:
$$c\cos\theta+R\theta\sin\theta+\frac{h}{2}\cos\theta > R+ \frac{h}{2}$$Using $\cos\theta \approx 1- \frac{\theta^2}{2}$ and $\sin\theta \approx \theta$, we can rewrite the above inequality as:
$$R\left(1-\frac{\theta^2}{2}\right)+R\theta(\theta)+\frac{h}{2}\left(1-\frac{\theta^2}{2}\right) > R + \frac{h}{2}$$Simplifying, we see that the angle $\theta$ cancels out and we are left with:
$$\boxed{R > \frac{h}{2}}$$
\end{solution}

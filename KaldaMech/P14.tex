\begin{solution}{hard}
\textbf{a)} Let us draw the free body diagram:
\begin{center}
\begin{asy}
size(6cm);
import olympiad;
draw((0,0)--(0.5,-2));
draw((0,0)--(0,-2), dotted);
label("$r$", (0.25, -2), S);
draw((0.5,-2)--(0,-2), dotted);
label("$\theta$", (0.05,-0.6));
dot((0.5,-2));
draw((0.5,-2)--(1.2,-2), arrow=Arrow(4));
label("$m\omega^2R$", (1.2,-1.85));
draw((0.5,-2)--(0.5,-2.5), arrow=Arrow(4));
label("$mg$", (0.32,-2.5));
label("$\ell$", (0.45,-1.2));
draw(anglemark((0,-2), (0,0), (0.5,-2)));
\end{asy}
\end{center}

We note by analysis of torques that in order for there to be a restoring torque, we must have:
\[mgr>m\omega^2 r\ell\cos\theta\]We use the small angle approximation $\cos\theta=1$ to then yield
\[g>\omega^2\ell\]Thus
\[\boxed{\omega^2<\frac{g}{\ell}}\]

\textbf{b)} Let the angles produced by the rods be $\varphi_1$ and $\varphi_2$ respectively. We then have the potential energy to be 
\[V=-mgl\cos\varphi_1-mgl(\cos\varphi_1+\cos\varphi_2)\]
Using the small angle approximation $\cos\varphi=1-\frac{\varphi^2}{2}$ gives us 
\[V=-mgl\left(1-\frac{\varphi_1^2}{2}\right)-mgl\left(1-\frac{\varphi_1^2}{2}+1-\frac{\varphi_2^2}{2}\right).\]
According to idea 19, we can remove all constants or non-quadratic terms
\begin{align*}
V_g = mgl\left(\varphi_1^2+\frac{\varphi_2^2}{2}\right) 
\end{align*}

Finding the potential energy produced by the centrigal force will have a similar approach.
\[V=\frac{1}{2}m(l\sin\varphi_1\omega^2)^2+\frac{1}{2}m(l(\sin\varphi_1+\sin\varphi_2)\omega)^2\]
Using the small angle approximation $\sin\theta=\theta$ we have
\[V=\frac{1}{2}m(l\varphi_1\omega^2)^2+\frac{1}{2}m(l(\varphi_1+\varphi_2)\omega)^2\]
\[V_c=\frac{1}{2}m\omega^2l^2(2\varphi_1^2+2\varphi_1\varphi_2+\varphi_2^2)\]
The total potential energy is then 
\[V = V_g + V_c = mgl\left(\varphi_1^2+\frac{\varphi_2^2}{2}\right) + \frac{1}{2}m\omega^2l^2(2\varphi_1^2+2\varphi_1\varphi_2+\varphi_2^2)\]
Rearranging this, gives us the quadratic
\[V = (mgl - m\omega^2 l^2)\varphi_1^2 + (m\omega^2 l^2)\varphi_1\varphi_2 + \frac{1}{2}(mgl - m\omega^2 l^2)\varphi_2^2\]
The equilibrium $\varphi_1 = \varphi_2 = 0$ is stable if it corresponds to the potential energy minimum, i.e, if the polynomial yields positive values for any departure from the equilibrium point; this condition leads to two inequalities. First, upon considering $\varphi_2 = 0$ (with $\varphi_2 \neq 0$) we conclude that the multiplier of $\varphi_1^2$ (or $mgl - m\omega^2 l^2$) has to be positive. Second, for any $\varphi_2 \neq 0$, the polynomial should be strictly positive, i.e. if we equate this expression to zero and consider it as a quadratic equation for $\varphi_1$, there should be no real-valued roots, which means that the discriminant should be negative. Thus, by looking at our discriminant we find that 
\[
(m\omega^2 l^2)^2 - 4(mgl - m\omega^2 l^2)\cdot \frac{1}{2}(mgl - m\omega^2 l^2) < 0\]
\[
m\omega^2 l^2 < \sqrt{2}(mgl - m\omega^2 l^2)\]
\[
\left(\frac{1}{\sqrt{2}} + 1\right)m\omega^2 l^2 < mgl\]
\[\boxed{\omega^2 < \frac{(2 - \sqrt{2})g}{l}}\]
\end{solution}

\begin{solution}{hard}
The hardest thing about this problem, as Kalda noted, was drawing a diagram. Here we provide a diagram for us to work with. Let O be the center of the hoop and A the center of the revolving shaft.
\begin{center}
\begin{asy}
import olympiad;
size(10cm);
label("N", (1.38, 1.3), NE);
label("$\mu N$", (0.25, 1.15), NW);
label("O", (0,0), W);
label("A", (0.2, 0.2), N);
label("$mg$", (0.7868, -4.3), S);
label("$\theta$", (0.7868, 0.5), SW);
label("$\alpha$", (-0.05, -0.6), SE);
label("$\beta$", (0.7, -2.3), N);

draw(circle((0,0), 1), linewidth(1.5pt));
filldraw(circle((0.2, 0.2), 0.72), gray(0.9));
dot((0,0));
dot((0.2,0.2));
draw((0.25, -0.954)--(0.7868, -3), linewidth(1.5pt));
filldraw(circle((0.7868,-3), 0.1), black);
draw((0.7868,-3)--(0.7868,-4.3), arrow=Arrow(4));
draw((0,0)--(0,-3), dashed);
draw((0,0)--(0.7868, -3));
draw((0.2,0.2)--(0.7868,-3), dotted);
draw((0.7868, 0.6172)--(0.7868, -3), dashed);
draw((0, 0)--(0.7868, 0.6172));
draw((0.7868, 0.6172)--(0.25, 1.15), arrow=Arrow(4));
draw((0.7868, 0.6172)--(1.38, 1.3), arrow=Arrow(4));
draw((0.7868, 0.6172)--(0.7868, 1.8), arrow=Arrow(4));
label("Q", (0.7868, 1.8), N);

pair A,B,C,D,E,F,G, H;
B = (0.7868, -3);
A = (0,0);
C = (0.2, 0.2);
D = (0, -3);
E = (0.7868, 2.1);
F = (-0.5, 1.9);
G = (0.7868, 0.6172);
H = (0.7868, -3);
markscalefactor=0.05;
draw(anglemark(E, G, F, 4));
draw(anglemark(A, G, H, 4));
markscalefactor=0.1;
draw(anglemark(D, A, B, 4));
draw(anglemark(G, H, A, 4));
\end{asy}
\end{center}
Let $Q$ be the vector sum of the friction and normal forces\footnote{The frictional force is not constant throughout the entire process of slipping however it is maximum (or $\mu N$) when the shaft is at equilibrium angle.},
\[Q=\sqrt{\mu^2N^2+N^2}=N\sqrt{\mu^2+1}\]because the system is in equilibrium, then the frictional force, $\mu N$, must be equal to $mg\sin\theta$. We also know by simple trigonometry that $\mu N=Q\sin\theta$. Therefore, because the sum of forces are zero we have,
\[\mu N=mg\sin\theta=N\sqrt{\mu^2+1}\sin\theta.\]We must now establish this relation in terms of $\beta$. One may look towards a torque analysis, however a more elegant mathematical approach is by the law of sines. We know by law of sines that
\[\frac{\sin\beta}{r}=\frac{\sin\theta}{r+\ell}\implies\sin\theta=\frac{(r+\ell)\sin\beta}{r}\]Substituting this in for $\sin\theta$ we find
\[\mu N=N\sqrt{\mu^2+1}\frac{(r+\ell)\sin\beta}{r}\]\[\sin\beta= \frac{r\mu}{(r+\ell)\sqrt{\mu^2+1}} \implies\boxed{\beta= \sin^{-1}\left(\frac{r\mu}{(r+\ell)\sqrt{\mu^2+1}}\right)}\]

\end{solution}

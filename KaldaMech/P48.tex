\begin{solution}{hard}
Let us direct the z axis upward (this will fix the signs of the angular momenta). We first attempt to find the initial angular momentum. We note that 
\[L = MvR + I\omega\]
Substituting in $I = \frac{2}{5}MR^2$ and $v = \omega/R$ gives us $L = \frac{7}{5}MvR$. In the $x$-axis, the sign of angular momentum is negative because of the right hand rule, and in the $y$-axis the sign of angular momentum is postive. This gives us,
\begin{align*}
L_x = -\frac{7}{5}Mv_{y_0}R\\
L_y = \frac{7}{5}Mv_{x_0}R
\end{align*}
The ball will continue to move in the same velocity in the $y$-direction as no non-conservative forces are acting in the horizontal direction. In the $x$-axis, the ball will have a final velocity of $u$, which implies that the final angular momentum is 
\[L_{x_f} = -\frac{7}{5}Mv_y R - MuR\]
Setting this equal to the initial angular momentum because of conservation of angular momentum, we get 
\begin{align*}
-\frac{7}{5}Mv_{y} R&= -\frac{7}{5}Mv_y R - MuR\\
\frac{7}{5}v_{y_0}&= \frac{7}{5}v_y + u\\
v_y = v_{y_0} - \frac{5}{7}u
\end{align*}
This gives the final velocity to be $\boxed{(v_{x_0}, v_{y_0} - \frac{5}{7}u)}$.
\end{solution}

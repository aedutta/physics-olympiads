\begin{solution}{normal}
See problem 20, the hockey puck will travel in a straight line. Consider a differential piece located at $(x,y)$ and has a vertical component of velocity $v_y$. At a location $(-x,y)$, there will be a differential piece with a vertical component of velocity $-v_y$. Their horizontal components of velocity will be the same. Thus, if we change their horizontal components of velocity, say by pushing the puck rightwards, their vertical components will change in the same way such that they still cancel out to zero. As a result, since there is always going to be another point which cancels out the perpendicular component of velocity, the net force caused by friction in the perpendicular direction will be zero. The puck will travel in a straight line.
\vspace{2mm}

We can use the fact that the friction force at each point in the $\omega \neq 0$ case is not exactly opposite to the direction of translational motion. In the $\omega=0$ case, the friction force at each point is exactly opposite to the direction of motion. Furthermore, the magnitude of the friction force at each point is the same in both cases. Thus, the total friction force in the direction opposite to the direction of motion is less in the $\omega \neq 0$ case, so the translational acceleration is less, so it will move a longer distance.
\end{solution}
